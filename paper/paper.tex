\documentclass[smallabstract,smallcaptions]{dccpaper}

\usepackage[utf8]{inputenc}

\usepackage{amsmath}
\usepackage{amssymb}
\usepackage{graphicx}
\usepackage{textgreek}
\usepackage{url}

\newlength{\figurewidth}
\newlength{\smallfigurewidth}
\setlength{\smallfigurewidth}{2.75in}
\setlength{\figurewidth}{6in}


% Mathematics
\newcommand{\set}[1]{\ensuremath{\{ #1 \}}}
\newcommand{\abs}[1]{\ensuremath{\lvert #1 \rvert}}
\newcommand{\Oh}{\ensuremath{\mathsf{O}}}
\newcommand{\oh}{\ensuremath{\mathsf{o}}}
\newcommand{\Th}{\ensuremath{\mathsf{\Theta}}}

% Structures
\newcommand{\SA}{\textsf{SA}}
\newcommand{\ISA}{\textsf{ISA}}
\newcommand{\BWT}{\textsf{BWT}}
\newcommand{\FMI}{\textsf{FMI}}
\newcommand{\C}{\textsf{C}}
\newcommand{\RA}{\textsf{RA}}
\newcommand{\mSA}{\ensuremath{\mathsf{SA}}}
\newcommand{\mISA}{\ensuremath{\mathsf{ISA}}}
\newcommand{\mBWT}{\ensuremath{\mathsf{BWT}}}
\newcommand{\mC}{\ensuremath{\mathsf{C}}}
\newcommand{\mRA}{\ensuremath{\mathsf{RA}}}

% Operations
\newcommand{\LF}{\textsf{LF}}
\newcommand{\rank}{\textsf{rank}}
\newcommand{\select}{\textsf{select}}
\newcommand{\mLF}{\ensuremath{\mathsf{LF}}}
\newcommand{\mrank}{\ensuremath{\mathsf{rank}}}
\newcommand{\mselect}{\ensuremath{\mathsf{select}}}

% Shorthands
\newcommand{\Acoll}{\ensuremath{\mathcal{A}}}
\newcommand{\Bcoll}{\ensuremath{\mathcal{B}}}



\begin{document}

\title
{\large
\textbf{Burrows-Wheeler transform for terabases}
}


\author{%
Jouni Sirén \\[0.5em]
{\small\begin{minipage}{\linewidth}\begin{center}
\begin{tabular}{c}
Wellcome Trust Sanger Institute \\
Wellcome Trust Genome Campus \\
Hinxton, Cambridge, CB10 1SA, UK \\
\url{jouni.siren@iki.fi}
\end{tabular}
\end{center}\end{minipage}}
}


\maketitle
\thispagestyle{empty}


\begin{abstract}
Fooo
\end{abstract}


\Section{Introduction}

Sequence data

BWT, compression, FM-index, applications, repetitive, rle

ReadServer; multiple BWTs

Large scale construction: speed, space, hardware, efficiency

BWT construction: from SA \cite{Mori2008,Nong2011}
external SA \cite{Gonnet1992,Bingmann2013,Kaerkkaeinen2014a,Nong2014,Kaerkkaeinen2015a}
space efficient algorithms \cite{Hon2007,Kaerkkaeinen2007,Siren2009,Okanohara2009}
external BWT \cite{Ferragina2012,Beller2013}

Specialized for DNA: \cite{Bauer2013,Liu2014,Li2014a,Pantaleoni2014,Wang2015}

This paper: terabases


\Section{Background}

A \emph{string} $S[1,n] = s_{1} \dotsm s_{n}$ is a sequence of \emph{characters} over an \emph{alphabet} $\Sigma = \set{1, \dotsc, \sigma}$. For indexing purposes, we consider \emph{text} strings $T[1,n]$ terminated by an endmarker $T[n] = \$ = 0$ not occurring anywhere else in the text. \emph{Binary} sequences are strings over the alphabet $\set{0, 1}$. A \emph{substring} of string $S$ is a sequence of the form $S[i,j] = s_{i} \dotsm s_{j}$. Substrings of the type $S[1,j]$ and $S[i,n]$ are called \emph{prefixes} and \emph{suffixes}, respectively.

The \emph{suffix array} (\SA) \cite{Manber1993} is a simple full-text index of a string. Given a text $T$, its suffix array $\mSA_{T}[1,n]$ is an array of pointers to the suffixes of the text in \emph{lexicographic order}. (If the text is evident from the context, we will omit it and write just \SA.) We can build the suffix array in $\Oh(n)$ time using approximately $n(\log n + \log \sigma + 2)$ bits of space, or just $2n$ bits of working space over the text and the suffix array \cite{Nong2011}. Given a \emph{pattern} $P$, we can find the \emph{lexicographic range} $[sp,ep]$ of suffixes of the text prefixed by the pattern in $\Oh(\abs{P} \log n)$ time by binary searching in the suffix array. The corresponding range of pointers $\mSA[sp,ep]$ lists the \emph{occurrences} of the pattern in the text.

While the suffix array is simple, it requires several times more memory than the original text. For large texts, this can be a serious drawback. The more space-efficient alternatives to the suffix array are based on the \emph{Burrows-Wheeler transform} (\BWT) \cite{Burrows1994}, an easily reversible permutation of the text with a similar combinatorial structure to the suffix array. Given a text $T[1,n]$ and its suffix array, we can easily produce the \BWT{} as $\mBWT[i] = T[\mSA[i]-1]$ (with $\mBWT[i] = T[n]$, if $\mSA[i] = 1$).

A key observation is that if suffix $X$ precedes suffix $Y$ in lexicographic order, suffix $cX$ will also precede suffix $cY$, for any character $c$. Therefore, if $\mBWT[i]$ is the $j$\nobreakdash-th occurrence of character $c$ in the \BWT, and $\mSA[i]$ points to suffix $X$, suffix $cX$ is the $j$\nobreakdash-th suffix starting with $c$ in lexicographic order. This forms the essence of \LF\emph{\nobreakdash-mapping}. The suffix preceding $T[\mSA[i],n]$ in text order is the one starting at $\mSA[\mLF(i)]$, where
$$
\mLF(i) = \mC[\mBWT[i]] + \mBWT.\mrank(i, \mBWT[i]),
$$
$\mC[c]$ is the number of suffixes starting with a character smaller than $c$, and $S.\mrank(i,c)$ is the number of occurrences of character $c$ in the prefix $S[1,i]$. For searching, we use a more general definition of \LF\nobreakdash-mapping:
$$
\mLF(i,c) = \mC[c] + \mBWT.\mrank(i, c),
$$
which tells the number of suffixes $X$ of text $T$ with $X \le c T[\mSA[i],n]$ in lexicographic order. This is known as the \emph{lexicographic rank} $\mrank(cT[\mSA[i],n], T)$ of string $cT[\mSA[i],n]$ among the suffixes of text $T$.

The \emph{FM-index} (\FMI) \cite{Ferragina2005a} uses the Burrows-Wheeler transform as a full-text index by augmenting it with a few additional structures. The lexicographic range $[sp,ep]$ corresponding to pattern $P$ is found through \emph{backward searching}. Let $[sp_{i},ep_{i}]$ be the lexicographic range of suffixes of text $T$ matching suffix $P[i, \abs{P}]$ of the pattern. We can find $[sp_{i-1},ep_{i-1}]$ as
$$
[sp_{i-1},ep_{i-1}] = [\mLF(sp_{i}-1, P[i-1]) + 1, \mLF(ep_{i}, P[i-1])].
$$
By starting from $[sp_{\abs{P}}, ep_{\abs{P}}] = [\mC[P[\abs{P}]]+1, \mC[P[\abs{P}]+1]]$, we can find the lexicographic range of suffixes starting with the pattern in $\Oh(\abs{P} \cdot t_{r})$ time, where $t_{r}$ is the time required to answer \rank{} queries on the \BWT. In practice, the time complexity ranges from $\Oh(\abs{P})$ to $\Oh(\abs{P} \log n)$, depending on the encoding.

In order to find the occurrences of the pattern in the text, the FM-index \emph{samples} some suffix array pointers, and uses \LF\nobreakdash-mapping to derive the unsampled pointers. The set of sampled pointers always includes a pointer to the beginning of each text. If $\mSA[i]$ is not sampled, the FM-index proceeds to $\mLF(i)$ and continues from there. If $\mSA[\mLF^{k}(i)]$ is the first sample encountered, $\mSA[i] = \mSA[\mLF^{k}(i)] + k$. Depending on the way the samples are selected, we may need a binary sequence to mark the pointers that have been sampled.

Assume that we have an ordered \emph{collection} of texts $\Acoll = (T_{1}, \dotsc, T_{m})$ of total length $n = \abs{\Acoll} = \sum_{i} \abs{T_{i}}$. We want to build a (generalized) \BWT{} for the collection. The usual way is to assume that the endmarkers of all text are distinct, with the one at the end of text $T_{i}$ having character value $(0,i)$. This guarantees that each suffix has a unique lexicographic rank among the suffixes of the collection. We still encode each endmarker as $0$ in the \BWT{} in order to save space. Because of this, \LF\nobreakdash-mapping does not work with $c = 0$, and we cannot match patterns spanning text boundaries.

With large collections of short texts (such as short reads), there are more space-efficient alternatives to sampling. Because all endmarkers had distinct character values during sorting, we know that $\mSA[i]$ (with $i \le m)$ points to the endmarker of text $T_{i}$. To reach the endmarker, we iterate
$$
\Psi(i) = \mBWT.\mselect(i - \mC[c], c),
$$
where $c$ is the largest character value with $\mC[c] < i$ and $S.\mselect(i,c)$ finds the $i$\nobreakdash-th occurrence of character $c$ in string $S$. If $k \ge 0$ is the smallest value for which $j = \Psi^{k}(i) \le m$, we know that $\mSA[i]$ points to offset $\abs{T_{j}} - k$ in text $T_{j}$.

We can \emph{extract} text $T_{i}$ in $\Oh(\abs{T_{i}} \cdot t_{r})$ time by using \LF\nobreakdash-mapping \cite{Burrows1994}. We start from the endmarker at $\mBWT[i]$ and extract the text backwards as
$$
T_{i}[\abs{T_{i}} - j] = \mBWT[\LF^{j-1}(i)] \,\, \textrm{for} \,\, 1 \le j < \abs{T_{i}}.
$$


\Section{Space-efficient \BWT{} construction}

The FM-index was introduced as a more space-efficient alternative to the suffix array. If we have to build a suffix array first in order to construct an FM-index, we lose a large part of the benefits, and index construction becomes the overall bottleneck. To overcome this bottleneck, we can use \emph{incremental construction algorithms} that build the FM-index directly. Some of them use an adjustable amount of working space on top of the FM-index, making it possible to index text colletions that are too large to fit in memory in an uncompressed form.

Assume that we have built the \BWT{} of text $T$, and we want to \emph{transform} the \BWT{} into that of text $cT$, where $c$ is a character. The transformation \cite{Hon2007} is quite straightforward. We first find the pointer $\mSA[i]$ to the beginning of text $T$ (the position $i$, where $\mBWT[i] = 0$). Then we determine the lexicographic rank $j = \mrank(cT, T) = \mC[c] + \mBWT.\mrank(i, c)$ of text $cT$ among the suffixes of text $T$. Finally we transform the \BWT{} by \emph{replacing} the endmarker at $\mBWT[i]$ with the inserted character $c$, and by \emph{inserting} a new endmarker between $\mBWT[j]$ and $\mBWT[j+1]$.

There are six basic ways of using the transformation in a \BWT{} construction algorithm. Instead of inserting a single character at a time, we may use \emph{batch updates} and transform the \BWT{} of text $T$ directly into that of text $XT$, where $X$ is any string \cite{Hon2007}. We can start with the \BWT{}s of text collections $\Acoll$ and $\Bcoll$, and \emph{merge} them into the \BWT{} of collection $\Acoll \cup \Bcoll$ \cite{Siren2009}. We can also have the \BWT{} of a text collection, and \emph{extend} multiple texts by inserting a new character to the beginning of each of them in a single update \cite{Bauer2013}. In all three cases, we may use either \emph{static} or \emph{dynamic} \cite{Chan2007} structures for the \BWT.


\Section{Merging \BWT{}s of text collections}

% FIXME: a running example with 2 texts
Assume that we have two text collections $\Acoll$ and $\Bcoll$ of total length $n_{\Acoll}$ and $n_{\Bcoll}$, respectively, and that we have already built \BWT{}s for them. The algorithm for merging $\mBWT_{\Acoll}$ and $\mBWT_{\Bcoll}$ into $\mBWT_{\Acoll \cup \Bcoll}$ works in three phases \cite{Siren2009}:
\begin{enumerate}
\item \textbf{Search.} We search for all texts in collection $\Bcoll$ using $\mBWT_{\Acoll}$ and output the lexicographic rank $\mrank(X, \Acoll)$ for each suffix $X$ of collection $\Bcoll$. This takes $\Oh(n_{\Bcoll} t_{r})$ time. If we do not have collection $\Bcoll$ in plain form, we can extract the texts from $\mBWT_{\Bcoll}$ in the same asymptotic time.
\item \textbf{Sort.} We sort the ranks produced in the search phase to build the \emph{rank array} (\RA) of collection $\Bcoll$ relative to collection $\Acoll$. The rank array is defined as $\mRA_{\Bcoll \mid \Acoll}[i] = \mrank(X, \Acoll)$, where $X$ is the suffix pointed by $\mSA_{\Bcoll}[i]$. The array requires $n_{\Bcoll} \log n_{\Acoll}$ bits of space, and we can build it in $\Oh(sort(n_{\Bcoll}, n_{\Acoll}))$ time, where $sort(n, u)$ is the time required to sort $n$ integers from universe $[0,u]$.
\item \textbf{Merge.} We interleave $\mBWT_{\Acoll}$ and $\mBWT_{\Bcoll}$ according to the rank array. If $\mRA_{\Bcoll \mid \Acoll}[i] = j$, the merged \BWT{} will have $j$ characters from $\mBWT_{\Acoll}$ before $\mBWT_{\Bcoll}[i]$. This phase takes $\Oh(n_{\Acoll} + n_{\Bcoll})$ time and uses $\Oh(b)$ bits of working space, where $b$ is the size of virtual memory pages.
\end{enumerate}
The overall time complexity of \BWT{} merging is $\Oh(n_{\Acoll} + n_{\Bcoll} t_{r} + sort(n_{\Bcoll}, n_{\Acoll}))$. The algorithm uses $n_{\Bcoll} \log n_{\Acoll} + \Oh(b)$ bits of working space in addition to the \BWT{}s, the structures required to use them as FM-indexes, and the working space used by the sorting algorithm.

In an alternative approach, we build the \emph{interleaving bitvector} $B_{\Acoll \cup \Bcoll}$. The bitvector is a binary sequence of length $n_{\Acoll} + n_{\Bcoll}$ and encodes the same information as the rank array. We set $B_{\Acoll \cup \Bcoll}[i + \mRA_{\Bcoll \mid \Acoll}[i]] = 1$ for $1 \le i \le n_{\Bcoll}$. If we extract the texts from $\mBWT_{\Bcoll}$ in the search phase, this can be done directly using the information available at that point. This approach uses $\Oh(n_{\Acoll} + n_{\Bcoll} t_{r})$ time and $n_{\Acoll} + n_{\Bcoll} + \Oh(b)$ bits of working space.


\Section{Large-scale \BWT{} merging}

main problem: memory size may be $\oh(\min(n_{\Bcoll} \log n_{\Acoll}, n_{\Acoll} + n_{\Bcoll}))$ bits even if a run-length encoded \BWT{} fits in memory; space overhead from dynamic representations also too large

search: trie vs individual sequences, multiple threads

buffering: run buffer, thread buffer (RLE, differential encoding, dynamic paged array), merge buffers, to disk

interleaving: multiway merge from disk in one thread, interleave BWTs in another, dynamic paged array


\Section{Experiments}

implementation, environment

data: ReadServer BWTs, NA12878 trio

ReadServer: trade-offs with different parameters

trio: ropebwt + this vs ropebwt vs ropebwt2


\Section{Discussion}

conclusions

sequence ordering: RLO/RCLO for compression, RLCO for mapping between reverse complements, mapping position for encoding pairing information

remove duplicates


\Section{References}
\bibliographystyle{IEEEtran}
\bibliography{paper}


\end{document}
